\documentclass[a4paper,12pt]{article}
\usepackage{mystyle}

\title{QM study}
\author{yongjun Kwon}
\date{\today}

\begin{document}

\maketitle
\vspace{-2em} % 간격을 줄이기 (음수 값: 줄임)

\setcounter{section}{1}

\section*{11.4 Parity Invariance}
\setheaders{\textit{QUANTUM MECHANICS}}{\raisebox{0.5ex}{\tiny$\square$}}
\setfooters{\textsc{Chapter 11.4 Parity Invariance}}{\thepage}

Parity is a discrete transformation. It corresponds to reflecting to the origin.
\begin{align*}
  x  \underset{\text{parity}}{\longrightarrow} -x \\
  p \underset{\text{parity}}{\longrightarrow} -p
\end{align*}

In quantum theory, parity operator acts like,
\[
\Pi | x \rangle = | -x \rangle
\]
you can easily check how it's done on arbitrary state \(\langle \psi | \)
\begin{align*}
  \Pi | \psi \rangle  &=\Pi \int_{\infty}^{-\infty } | x \rangle\langle x | \,dx  \\
    &= \int_{\infty}^{-\infty} | -x \rangle\langle x |  \,dx \\
    &= \int_{\infty}^{-\infty} | x' \rangle\langle -x' | \psi \,dx'  
\end{align*}

So the result is
\[
| x \rangle\langle \psi | = \psi (x) 
\]
\[
\langle x | \Pi | \psi \rangle =\psi(-x)
\]

If you use the realation of \(\langle x | p \rangle  = \frac{1}{\sqrt{2 \pi \hbar}} e^{\frac{ipx}{\hbar}} \), you can easily see that

\[
\Pi | p \rangle =| -p \rangle \qquad (\text{with out phase factor})
\]

\begin{proposition}[Parity operator $\Pi$] \hfill \\
  1. $\Pi = \Pi^{-1}$. \\
  2. The eigenvalues are $\pm 1 \qquad (\Pi^2= I)$.\\
  3. So $\Pi$ is Hermitian and unitary.\\
  4. So $\Pi^{-1}=\Pi^\dagger=\Pi$.
\end{proposition}

Eigenvalues can determine that eigenvectors are even/odd. In x, p basis form, you can think about its wave function is parity even/odd also.
But if you know that particular vector is even/odd, it does not mean that it have a even/odd wave function in arbitrary basis form.
Means \(\psi(\omega) \underset{\text{parity}}{\longrightarrow} \psi(-\omega)\) doesn't always true.

We may define operators as Heigenberg way.
\begin{align*}
  \Pi^\dagger X \Pi = -X \\
  \Pi^\dagger P \Pi = -P \\
\end{align*}

Then Hamiltonian Parity invariant if
\[
\Pi^\dagger H(X,P) \Pi = H(-X,-P) = H(X,P)
\]
If you simply multiply both sides you can get
\[
[\Pi,H]=0 
\]
In particular if we consider only one dimension, we can show that eigenfunctions of harmonic oscillator and infinte potential well problem are also eigenfunction of
Parity operator.
\begin{remark}[infinte box potential]
  Two cases for it. \(-L/2 ~ L/2\) case have potential symmetric under parity but \(0~L\) don't. So \(0~L\) case
  have no definite parity.
\end{remark}

Because Hamiltonian and U(t) commutes, if parity operator commutes with Hamiltonian, it also commutes with U(t).
\[
\Pi U(t)=U(t)\Pi
\]
It means that the operation of applying parity followed by time evolution for
t is equivalent to applying time evolution for
t followed by parity. \\
\indent One famous example of Parity non-Invariance is weak force. The famous experiment is well described on textbook pg.299-300.
\newpage

\section*{11.5 Time-Reversal Symmetry}
\setheaders{\textit{QUANTUM MECHANICS}}{\raisebox{0.5ex}{\tiny$\square$}}
\setfooters{\textsc{Chapter 11.5 Parity Invariance}}{\thepage}

Time reversed state can be thought as one in which the position is same but the momentum is reversed.
Think the earth moves counter-clockwise from \(\theta=0\) at \(t=0\). When \(t=T\), superman reversed its motion as
clockwise, then we can see after the same amount of time flies that:
\[
x(2T)=x(0) \qquad p(2T)=-p(0)
\]
(I didn't care about vector here for simplification). The above defines the TRI(Time-Reversal Invariance.). \\
\indent Let's check out how time reversal comes out from Newton's law.
If we consider t=0 defines the point when the motion is time-reversed,
\[
x_r (t)=x(-t)
\]
The reversal of velocities follow from this:
\[
\dot{x_r}(t)=\frac{\mathrm{d} x(-t)}{\mathrm{d} t}=-\frac{\mathrm{d} x(-t)}{\mathrm{d} (-t)}=-\dot{x}(-t)
\]
You can see time reversed trajectory also obeys newton's second law:
\[
\frac{d^2 x_r(t)}{d t^2}=m \frac{d^2 x(-t)}{d t^2}=m \frac{d^2 x(-t)}{d(-t)^2}=F(x(-t))=F\left(x_r(t)\right)
\]
So the reversed motion follows the newton's second law like as x(t) does.\\
\indent What is the example of non time-reversal invariant? We can think of a charged particle that moves through
the +z magnetic area. Since velocity is related with Lorentz force, it does not come back to original position
if you reverse its velocity at \(t=T\). 

\indent How does it appear at quantum mechanics? The answer is complex conjugate:
\[
\psi \rightarrow \psi^*
\]
Consider time independent shrodinger equation on x-basis. 
\[
i \hbar \frac{\partial \psi(x, t)}{\partial t}=H(x) \psi(x, t)
\]
Complex conjugate did not influence on probability distribution by x. But in plane wave, complex conjugate give us 
reversed momentum.\footnotemark
\footnotetext{I'm concerning about it is okay to think that applying complex conjugate on $-i\hbar \frac{\partial}{\partial x}$
is Heigenberg picture. What do you think?}
\\
\indent If the system has TRI, We can think about following steps:
\[
\psi(x, 0) \rightarrow e^{-i H(x) T / \hbar} \psi(x, 0) \rightarrow e^{i H^*(x) T / \hbar} \psi^*(x, 0) \rightarrow e^{-i H(x) T / \hbar} e^{i H^*(x) T / \hbar} \psi^*(x, 0)
\]
In order to acheive TRI, we require that
\[
H(x)=H^*(x)
\]
then
\[
\psi(x, 2T)=\psi^*(x,0)
\]
ie, Hamiltonian be real. \(H=\frac{P^2}{2m}+V(x)\) meets this criteria even in higher dimension. But, if we have a 
magnetic field, P enters linearly and \(H(x)=H^*(x)\) breaks. Also note that if Hamiltonian is real, every eigenfunction 
implies a degenerate one, which is its conjugate.
\begin{remark}[magnetic field]
  Failure of TRI due to the magnetic field does not imply that E.M is fundamentally breaks the time-reversal symmetry.
  If you incoporate the currnets that make magnetic field to your whole system, E.M meets the time-reversal symmetry.
\end{remark}



\end{document}